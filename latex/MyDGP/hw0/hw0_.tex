\documentclass[a4paper]{D:/MyRepo/latex/PaperReadingLog}
\usepackage{caption}
\usepackage{overpic}
\usepackage{float}

\usepackage{enumitem} 
\usepackage{amsfonts,amssymb}
\usepackage[linesnumbered,ruled,lined]{algorithm2e}%%伪代码
\usepackage{multicol}

\begin{document}

\PaperInfo
{CGAL使用记录}
{朱}
{天宇}
{}
\section{windows下安装cgal}
\subsection{使用vcpkg安装}
按照\href{https://doc.cgal.org/latest/Manual/windows.html}{官网指示},推荐使用vcpkg来安装cgal。

但是vcpkg在国内会产生网络连接问题,所以还是使用传统方案。(悲)

\subsection{使用源码编译cgal的qt版本}
\begin{enumerate}
    \item 指令为 git clone https://github.com/CGAL/cgal \#下载cgal源码,由于未知原因,很慢
    \item 下载下来的源码是什么不清楚,转回曾经使用的installer进行安装(悲)
\end{enumerate}

\subsection{使用installer安装}
\begin{enumerate}
    \item 下载\href{https://github.com/CGAL/cgal/releases/download/v5.1.2/CGAL-5.1.2-Setup.exe}{CGAL-5.1.2-Setup.exe}
    \item 在D:\\LIBRARY\\cgal目录下安装
    \item 在用exe安装的时候,installer会尝试配置系统环境变量,提示手动设置为安装的路径。如下图\begin{figure}[H]%使用H表示Here,不对图片排版!
        \centering
        \begin{overpic}[width=0.66\linewidth]{fig/1.png}
        \end{overpic}
        \vspace{-3.5mm}
        \caption{提示配置路径}
        \vspace{2mm}
    \end{figure}
    \item 新建文件夹build,cd build, cmake-gui打开cmake图形界面,\textbf{一定要用图形界面}
    \item 注意QT路径有没有配好,之后configure+generate
    \item 生成了一个CGAL.sln,双击打开,选择Release+x64,重新生成
    \item 注意Cgal的文件夹下的INSTALL.md文件,里面说从5.0版本开始,cgal变成一个header-only的库,也就是说不用build……
\end{enumerate}

\section{启动cgal的一个官方example}
按照\href{https://blog.csdn.net/a825346034/article/details/103759448}{这里}的说明,打开examples\\Surface\_mesh,新建build文件夹后cmake ..创建项目,编译其中的draw\_surface\_mesh子项目,设定好读入的off文件(不支持obj),随后启动程序。可以看到弹出如下的窗口。

\begin{figure}[H]%使用H表示Here,不对图片排版!
    \centering
    \begin{overpic}[width=0.5\linewidth]{fig/2.png}
    \end{overpic}
    \vspace{-3.5mm}
    \caption{生成的效果}
    \vspace{2mm}
\end{figure}

\section{使用cmake来对CGAL进行配置}
测试项目的位置是D:/MyRepo/DigitalGeometryPrecessing/MyDGP/build

cmake通过\textbf{find\_package(CGAL)语句}来寻找CGAL。测试项目需要画图,需要在cmakelists.txt的写法参考了cgal/examples/Surface\_mesh文件夹中的cmakelists写法,包含了find\_package(CGAL)以及预处理器定义、配置QT等内容。

生成的release版本没有pdb文件无法运行,这似乎是cmake生成qt项目的通病了,可以用qt提供的windeployqt.exe打包一下,或者在cmake中添加自动打包的相关代码。

\section{配置Eigen3}
Eigen除了C++标准库之外没有任何依赖。但是直接解压的话,cmake会找不到Eigen,需要手动在cmakelists.txt中添加,这比较麻烦。如果使用cmake .. 指令的输出项目中提示Could NOT find Eigen3,那么只要\href{https://blog.csdn.net/weixin\_42451919/article/details/103444420}{使用cmake-gui手动指定eigen目录即可。}



% \begin{center}
%     \fcolorbox{white}{gray!10}{\parbox{.9\linewidth}{Test Test Test Test Test Test Test Test Test Test Test Test Test Test Test Test Test Test Test Test Test Test Test Test Test Test Test Test Test Test Test Test.}}
%     \end{center}
\end{document}
%去掉下划线,采用大写字母